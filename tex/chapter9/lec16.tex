\documentclass[../main.tex]{subfiles}
\begin{document}
\lecture{16}{08.04}{}
\begin{theorem}
    $E$ - ограниченное замкнутое множество $E\subset \E_{n}$; $f(x)$ непрерывна на $E$ по $E$ $\implies f(x)$ ограничена на $E$.
\end{theorem}
\begin{proof}
    $f(x)$ неограничена на $E$, т.е $\forall M>0 \; \exists x \in E : |f(x)|>M$.\\ $M=M_{m}=m=1,2,\dots \quad \exists x^{(m)}\in E: |f(x^{(m)})|> M_{m}=m.$
    \\$\{x^{(m)}\}_{m=1}^{\infty} \implies \exists \{ x^{(m_{p})}\}\subset \{x^{(m)}\}: \exists \lim\limits_{p  \to \infty} x^{(m_{p})}=x^{(0)} \implies x^{(0)}$ - точка прикосновения $E\implies x^{(0)}\in E$.\\ 
    $\lim\limits_{p \to \infty}f(x^{(m_{p})})=f(x^{(0)}),|f(x^{(m_{p})})>m_{p}\geqslant 0 \implies \lim\limits_{p   \to \infty}f(x^{(m_{p})}) =\infty $
\end{proof}
$E$ - ограниченное замкнутое множество, $f(x)$ - непрерывная на $E$ по $E$ $\implies \exists M < +\infty, \exists m> -\infty: \underset{x \in E}{\sup}{f(x)}=M, \underset{x\in E}{\inf}{f(x)}=m, m\leqslant M. $ 
\begin{theorem}
    $E\subset \E_{n}, E$ - ограниченное замкнутое множество, $f(x)$ - непрерывная на $E$ по $E$, $m = \underset{x\in E}{\inf}f(x), M = \underset{x\in E}{\sup}f(x) \implies \exists x^{(1)},x^{(2)} \in E : f(x_{1})=m, f(x_{2})=M.$
\end{theorem}
\begin{proof}
    Самостоятельно. 
\end{proof}
\begin{definition} 
    $E \subset \E_{n}$, $f(x)$ называется равномерно непрерывной на $E$, если $\forall \varepsilon > 0 \; \exists \delta > 0 :\\ \forall x,y \in E, \rho(x,y) = \| x-y\| < \delta \implies |f(x)-f(y)|<\varepsilon.$
\end{definition}
\begin{theorem}
    $E\subset \E_{n}$, $f(x)$ равномерно непрерывна на $E$ $\implies f(x)$ непрерывна на $E$ по $E$.
\end{theorem}
\begin{proof}
    Самостоятельно.
\end{proof}
\begin{theorem}
    $E\subset \E_{n}$, $E$ - ограниченное замкнутое множество, $f(x)$ - непрерывна на $E$ по $E\implies$ $f(x)$ равномерно непрерывна на $E$.
\end{theorem}
\begin{proof}
    Самостоятельно.
\end{proof}

\section{Производные и дифференциалы функций нескольких переменных}
\begin{definition}
    $x^{(0)} = (x_{1}^{(0)},\dots,x_{n}^{(0)})\in \E_{n}$, $f(x)$ определена в окрестности точки $x^{(0)}$. Частной производной по переменной $x_{i}(i=1,2,\dots,n)$ при $x=x^{(0)}$ (в точке $x^{(0)}$) функции $f(x)$ называется $\frac{\partial{f}}{\partial{x_{i}}}(x^{(0)})= \\=\frac{\partial{f}}{\partial{x_{i}}}(x_{1}^{(0)},\dots,x_{i}^{(0)},\dots,x_{n}^{(0)})\equiv\lim\limits_{\Delta x_{i}\to 0}\frac{f(x_{1}^{(0)},\dots,x_{i-1}^{(0)},x_{i}^{(0)},x_{i+1}^{(0)},\dots,x_{n})-f(x_{1}^{(0)},\dots,x_{i}^{(0)},\dots,x_{n}^{(0)})}{\Delta x_{i}} $\\     
\end{definition}
\noindent$\frac{\partial{f}}{\partial{x_{i}}}(x^{(0)})\bigg|_{\text{пр}}; \frac{\partial{f}}{\partial{x_{i}}}(x^{(0)})\bigg|_{\text{лев}}\quad \frac{\partial^{2}{f}}{\partial{x_{i}}\partial{x_{i}}}(x^{(0)})=\frac{\partial{u}}{\partial{x_{i}}}(x^{(0)}), u( x )=\frac{\partial{f}}{\partial{x_{i}}}(x)\quad \frac{\partial^{3}{f}}{\partial{x_{i}}\partial{x_{i}}\partial{x_{i}}}(x^{(0)})= \frac{\partial{v}}{\partial{x_{i}}}(x^{(0)}), \\v (x) = \frac{\partial^{2}{f}}{\partial{x_{i}}\partial{x_{i}}}(x),\dots$
\vspace{0.2cm}

$\frac{\partial^{2}{f}}{\partial{x_{i}}\partial{x_{j}}}(x^{(0)})=\frac{\partial{w}}{\partial{x_{i}}}(x^{(0)}),w(x ) = \frac{\partial{f}}{\partial{x_{j}}}(x)$

Рассмотрим $f(x,y) = \begin{cases}
    xy \frac{ x^{2}-y^{2}}{x^{2}+y^{2}}, x^{2}+y^{2} > 0 \\ 
    0, x=y=0
\end{cases}$\\ 
$(x,y)\neq (0,0) \quad \frac{\partial{f}}{\partial{x}}(x,y)= y \frac{x^{2}-y^{2}}{x^{2}+y^{2}}+ xy \frac{2x(x^{2}+y^{2})-2x(x^{2}-y^{2})}{(x^{2}+y^{2})^{2}}= y\left[\frac{x^{2}-y^{2}}{x^{2}+y^{2}}+ \frac{4x^{2}y^{2}}{(x^{2}+y^{2})^{2}}\right]; \\ \frac{\partial{f}}{\partial{x}}(0,y)=-y (y\neq 0)$ \qquad 
$\frac{\partial{f}}{\partial{x}}(0,0) = \lim\limits_{\Delta x \to 0}\frac{f(0+\Delta x ,0)-f(0,0)}{\Delta x} = 0\implies \frac{\partial{f}}{\partial{x}}(0,y)=-y\;\forall y$. \\ 
Найдем $\frac{\partial^{2}{f}}{\partial{y}\partial{x}}(0,0)=\lim\limits_{\Delta y\to 0}\frac{\frac{\partial{f}}{\partial{x}}(0,0)-\frac{\partial{f}}{\partial{x}}(0,0)}{\Delta y} = \lim\limits_{\Delta y\to 0} \frac{ - \Delta y -0}{\Delta y} = -1.  $ 
\\Пусть $(x,y)\neq (0,0)\quad \frac{\partial{f}}{\partial{y}}(x,y) = x\left[\frac{x^{2}-y^{2}}{x^{2}+y^{2}}- \frac{4x^{2}y^{2}}{(x^{2}+y^{2})^{2}}\right]; \frac{\partial{f}}{\partial{y}}(x,0) = x (x\neq 0); \frac{\partial{f}}{\partial{y}}(0,0)=\\=\lim\limits_{ \Delta y\to 0}\frac{f(0,0+\Delta y)-f(0,0)}{\Delta y}=0, \frac{\partial{f}}{\partial{y}}(x,0) = x \; \forall x$. 
Тогда $\frac{\partial^{2}{f}}{\partial{x}\partial{y}}(0,0)= \lim\limits_{\Delta x\to 0}\frac{\frac{\partial{f}}{\partial{y}}(0+\Delta x,0)- \frac{\partial{f}}{\partial{y}}(0,0)}{\Delta x}=1 $
\vspace{0.5cm}
\begin{theorem}
    Пусть $f(x):$ в окрестности точки $x^{(0)}$ $\exists \frac{\partial{f}}{\partial{x_{i}}}(x), \frac{\partial{f}}{\partial{x_{j}}}(x), \frac{\partial^{2}{f}}{\partial{x_{i}}\partial{x_{j}}}(x), \frac{\partial^{2}{f}}{\partial{x_{j}}\partial{x_{i}}}(x), \frac{\partial^{2}{f}}{\partial{x_{i}\partial{x_{j}}}}$ и $\frac{\partial^{2}{f}}{\partial{x_{j}}\partial{x_{i}}}$ непрерывны при $x=x^{(0)}\implies \frac{\partial^{2}{f}}{\partial{x_{i}}\partial{x_{j}}}(x^{(0)})=\frac{\partial^{2}{f}}{\partial{x_{j}}\partial{x_{i}}}(x^{(0)})$.
\end{theorem}
\begin{proof}
    Рассмотрим $W = f(x_{i}^{(0)}+\Delta x_{i}, x_{j}^{(0)}+\Delta x_{j}) - f(x_{i}^{(0)}+\Delta x_{i}, x_{j}^{(0)}) - f(x_{i}^{(0)}, x_{j}^{(0)}+\Delta x_{j}) +\\+ f(x_{i}^{(0)}, x_{j}^{(0)})$. Введем две функции от одной переменной $g_{1}(x_{i}) = f(x_{i}, x_{j}^{(0)}+\Delta x_{j}) - f(x_{i}, x_{j}^{(0)})$ в окрестности точки $x_{i}^{(0)}$ и $g_{2}(x_{j}) = f(x_{i}^{(0)}+\Delta x_{i}, x_{j}) - f(x_{i}^{(0)}, x_{j})$ в окрестности точки $x_{j}^{(0)}$.\\ Тогда $W = g_{1}(x_{i}^{(0)}+\Delta x_{i}) - g_{1}(x_{i}^{(0)}) = g_{2}(x_{j}^{(0)}+\Delta x_{j}) - g_{2}(x_{j}^{(0)})$.
    \\По теореме Лагранжа: $g_{1}' (x_{i}^{(0)}+\theta_{1} \Delta x_{i})\Delta x_{i} = g_{2}'(x_{j}^{(0)}+\theta_{2} \Delta x_{j})\Delta x_{j}; \quad g_{1}'(x_{i}^{(0)} + \theta_{1} \Delta x_{i}) = \frac{\partial{f}}{\partial{x_{i}}}(x_{i}^{(0)}+\theta_{1} \Delta x_{i}, x_{j}^{(0)}+\Delta x_{j}) -  \frac{\partial{f}}{\partial{x_{i}}}(x_{i}^{(0)}+\theta_{1} \Delta x_{i}, x_{j}^{(0)}) =$ Т. Лагранжа $= \frac{\partial^{2}{f}}{\partial{x_{j}}\partial{x_{i}}}(x_{i}^{(0)}+\theta_{1} \Delta x_{i}, x_{j}^{(0)}+\theta_{3} \Delta x_{j})\Delta x_{j} ;
    \\ g'_{2}(x_{j}^{(0)}+\theta_{2} \Delta x_{j}) = \frac{\partial{f}}{\partial{x_{j}}}(x_{i}^{(0)}+\Delta x_{i }, x_{j}^{(0)}+\theta_{2} \Delta x_{j}) - \frac{\partial{f}}{\partial{x_{j}}}(x_{i}^{(0)}, x_{j}^{(0)}+\theta_{2} \Delta x_{j}) = \frac{\partial^{2}{f}}{\partial{x_{i}}\partial{x_{j}}}(x_{i}^{(0)}+\theta_{4} \Delta x_{i}, x_{j}^{(0)}+\theta_{2} \Delta x_{i})\Delta x_{i}$.\\
    $\frac{\partial^{2}{f}}{\partial{x_{j}}\partial{x_{i}}}(x_{i}^{(0)}+\theta_{1} \Delta x_{i}, x_{j}^{(0)}+\theta_{3} \Delta x_{j})= \frac{\partial^{2}{f}}{\partial{x_{i}}\partial{x_{j}}}(x_{i}^{(0)}+\theta_{4} \Delta x_{i}, x_{j}^{(0)}+\theta_{2} \Delta x_{j})$\\
\end{proof}
\begin{definition}
    $x^{(0)}\in \E_{n}$, $f(x)$ определена в окрестности точки $x^{(0)}$, $\omega \in \E_{n}, \| \omega \| =1.$ Производной функции $f(x)$ по направлению $\omega$ в точке $x^{(0)}$ называется $\frac{\partial{f}}{\partial{\omega}}(x^{(0)}) \equiv \lim\limits_{t \to 0+0} \frac{f(x^{(0)}+t\omega)-f(x^{(0)})}{t}$
\end{definition}
\vspace{0.5cm}
Пример: $\omega = (0,\dots,0,\overset{i}{1},\dots,0) \quad \frac{\partial{f}}{\partial{\omega}}(x^{(0)}) = \lim\limits_{t\to 0+0} \frac{f(x_{1}^{(0)},\dots,x_{i}^{(0)}+t,\dots,x_{n}^{(0)})-f(x^{(0)})}{t} = \frac{\partial{f}}{\partial{x_{i}}}(x^{(0)})\bigg|_{\text{пр}}$
\\Проверить самостоятельно: $\omega =(0,\dots,0,\overset{i}{-1},\dots,0) \quad \frac{\partial{f}}{\partial{\omega}}(x^{(0)}) = - \frac{\partial{f}}{\partial{x_{i}}}(x^{(0)})\bigg|_{\text{лев}}$
\vspace{1cm}

Пример: рассмотрим $f(x,y) = \begin{cases}
    \sqrt{x^{2}+y^{2}} \sin{\frac{1}{\sqrt{x^{2}+y^{2}}}},(x,y) \neq (0,0) \\
    0, (x,y) = (0,0)
\end{cases}$\\
\vspace{0.2cm}
$|f(x,y)|\leqslant \sqrt{x^{2}+y^{2}}\implies \underbrace{-\sqrt{x^{2}+y^{2}}}_{\to 0}\leqslant f(x,y)\leqslant \underbrace{\sqrt{x^{2}+y^{2}}}_{\to 0}$
\\ $\frac{\partial{f}}{\partial{\omega}}(0,0), \omega = (\cos{\alpha},\sin{\alpha})= \lim\limits_{t \to 0+0} \frac{f(t\omega)-f(0,0)}{t} = \lim\limits_{t \to 0+0} \frac{t\sin{\frac{1}{t}}}{t} =\lim\limits_{t\to 0+0} \sin{\frac{1}{t}}$ - не существует. $\omega = (\cos{\alpha},\sin{\alpha})$

Рассмотрим $f(x,y) = \begin{cases}
    1, y= x^{2} ,x\neq 0 \\
    0, \text{ в остальных точках}
\end{cases} \qquad \frac{\partial{f}}{\partial{\omega}}(0,0)=0\; \forall \omega = (\cos{\alpha},\sin{\alpha})$ 

