\documentclass[../main.tex]{subfiles}
\begin{document}
\newpage
\lecture{4}{11.02}{}
\section{Интегрируемость непрерывной функции. Интегрируемость монотонной функции}
\begin{theorem}
    $f(x)\in C[a,b]\implies f(x) \text{ инт на } [a,b]$
\end{theorem}

\begin{proof}
    $f(x)\in C[a,b]\underset{\text{т.Кантора}}{\implies} f(x)$ равномерно непрерывна на $[a,b],$ т.е. $\forall \varepsilon>0 \exists \delta>0 \forall x',x''\in[a,b]: |x'-x''|<\delta \implies |f(x')-f(x'')|<\frac{\varepsilon}{b-a}.$\\ 
    Берем $\forall T=\{a=x_{0}<x_{1}<\dots<x_{n}=b\}.\quad \delta_{T}<\delta;$\\ 
    $f(x)\in C[x_{k-1},x_{k}]\underset{\text{т. Вейрштрасса}}{\implies} \exists x'_{k},x''_{k}\in[x_{k-1},x_{k}]: \begin {aligned} M_{k}\underset{x_{k-1}\leqslant x_{k}}{=}supf(x)=f(x'_{k}) \\ 
    m_{k}=inf(f(x))=f(x''_{k})\end{aligned}.\qquad |x'_{k}-x''_{k}|\leqslant \Delta x_{k}\leqslant \delta_{t}\implies |M_{k}-m_{k}| < \frac{\varepsilon}{b-a}.$\\ 
    $0\leqslant M_{k}-m_{k}\leqslant \frac{\varepsilon}{b-a} | \Delta x_{k} \text{и} \sum_{k=1}^{n}\implies_{0}\leqslant S_{T}(f)-s_{T}(f)<\varepsilon\implies \lim\limits_{\delta_{T}\to 0}(S_{T}(f)-s_{T}(f))=0 \underset{\text{Т3}}{\implies}f(x) \text{ инт на } [a,b] $
    
\end{proof} 

\begin{theorem}
    $f(x)$ монотонна на $[a,b](\text{не имеет значения, что из себя представляет множество точек разрыва})\implies f(x) \text{ инт на } [a,b]$
    
\end{theorem}(

\begin{proof}
    Пусть $f(x)$ монотонно возрастает на $[a,b]\implies f(a)\leqslant f(x)\leqslant f(b) \forall x\in[a,b]\implies f(x) \text{ ограничена на } [a,b].$\\ 
    Берем $\forall T=\{a=x_{0}<x_{1}<\dots<x_{n}=b\}\qquad f(x_{k-1})\leqslant f(x)\leqslant f(x_{k}) \forall x\in [x_{k-1},x_{k}]\implies \begin{aligned} M_{k}\underset{x_{k-1}\leqslant x\leqslant x_{k}}=supf(x_{k})\\ m_{k}\underset{x_{k-1}\leqslant x\leqslant x_{k}}{=}inff(x)=f(x_{k-1})\end{aligned}\\$
    $0\leqslant S_{T}(f)-s_{T}(f)=\sum_{k=1}^{n}(M_{k}-m_{k})\Delta x_{k}\leqslant \delta_{T}\sum_{k=1}^{n}(M_{k}-m_{k})=\delta_{T}\sum_{k=1}^{n}(f(x_{k})-f(x_{k-1}))=\delta_{T}(f(b)-f(a))\underset{\delta_{T}\to 0}{\to} 0 \implies \lim\limits_{\delta_{T} \to 0}(S_{T}(f)-s_{T}(f))=0 \underset{\text{критерий инт. Т3}}{\implies} f(x) \text{ инт на }[a,b]$
    \\ Самостоятельно рассмотреть случай монотонного убывания.
\end{proof}

Пример. $f(x)=\begin{cases}\frac{1}{k},x\in(\frac{1}{k+1},\frac{1}{k}],k\in\mathbb{N}\\0,x=0\end{cases}$\\ 
$f(x) \infty$-но много точек разрыва на $[a,b]:x=\frac{1}{k},k=2,3,4,\dots -$ точки разрыва 1-го рода)\\ 
$f(x)$ монотонно возрастает на $[0,1]\implies f(x)$ инт на $[a,b]$

\section{Интегрируемость функции, отличающейся от интегрируемой в конечном количестве точек} 
\begin{theorem}
    Пусть $f(x)$ инт на $[a,b]\implies \tilde{f(x)}=\begin{cases}
        A, x=\tilde{x}\in[a,b]\\ 
        f(x),x\in[a,b] \text{искл}  \{\tilde{x}\}
    \end{cases}$ тоже инт на $[a,b],$ причем $\int _{a}^{b}f(x)dx=\int _{a}^{b}\tilde{f(x)}dx$
\end{theorem}
\begin{proof}
    $f(x)$ инт на $[a,b]\implies f(x)$ ограничена на $\begin{aligned}1)[a,b],\text{ т.е } \exists M>0:|f(x)|\leqslant M \forall x\in[a,b] \\ 2)\lim\limits_{\delta_{T}\to 0}  S_{T}(f)=\int _{a}^{b}f(x)dx=I\end{aligned}$ \\ 
    $\forall \varepsilon>0 \exists \delta>0: \forall T: \delta_{T}<\delta \implies|S_{T}(f)-I|<\frac{\varepsilon}{2}.$\\
    Берем $\delta_{2}=\frac{\varepsilon}{4(M+|A|)}>0 \implies \exists \delta=min(\delta_{1},\delta_{2})>0.\qquad \text{ Берем } \forall T=\{a=x_{0}<x_{1}<\dots<x_{n}=b\}: \delta_{T}<\delta.$ \\ 
    $\begin{aligned}
        M_{k}=supf(x) \\ 
        \tilde{M_{k}}=sup\tilde{f(x)}, k=1,n
    \end{aligned}$. Рассмотрим $|S_{T}(f)-S_{T}(\tilde{f})|=|\sum_{k=1}^{n}(M_{k}-\tilde{M_{k}}\Delta x_{k})|\leqslant \delta_{T}*2 (M+|A|)<2\delta(M+|A|)\leqslant 2\delta_{2}(M+|A|)=\frac{\varepsilon}{2}$ \\ 
    Рассмотрим $|S_{T}(\tilde{f})-I|=|S_{T}(\tilde{f})-S_{T}(f)+S_{T}(f)-I|\leqslant |S_{T}(\tilde{f})-S_{T}(f)|+\underbrace{|S_{T}(f)-I|}_{<\frac{\varepsilon}{2}}<\varepsilon,\text{ т.е } \lim\limits_{\delta_{T}\to 0}(S_{T}(f)-I)=0\implies\lim\limits_{\delta_{T}\to 0}S(\tilde{f})=I$\\ 
    Аналогично: $\lim\limits_{\delta_{T}\to 0}s_{T}(\tilde{f})=I \implies \lim\limits_{\delta_{T}\to 0} (S_{T}(\tilde{f})-s_{T}(\tilde{f}))=0\implies \tilde{f(x)} $ инт на $[a,b]$. \\ 
    Т.к $\int \tilde{f(x)}dx=\lim\limits_{\delta_{T}\to 0}S_{T}(f)=\lim\limits_{\delta_{T}\to 0}S_{T}(\tilde{f})\implies \lim\limits_{\delta_{T}\to 0}(S_{T}(\tilde{f}))=\int _{a}^{b}\tilde{f(x)}dx=> \int _{a}^{b}\tilde{f(x)}dx=\int _{a}^{b}f(x)dx    $
\end{proof}

\begin{corollary}
    $f(x)$ инт на $[a,b]\implies \tilde{f(x)},$ отличающася от $f(x)$ в конечном количестве точек, тоже инт на $[a,b]$,причем $\int _{a}^{b}f(x)dx=\int _{a}^{b}\tilde{f(x)}dx$
\end{corollary}
\begin{proof}
    Применим последнюю теорему надлежащее число раз.
\end{proof}
Пример. $\chi(x)=\begin{cases}
    1,x-\text{рац}\\ 
    0,x-\text{иррац}
\end{cases}$ отличающаяся от $f_{0}(x)\equiv 0$ на $[a,b]$ в счетном количество точек, но при этом $\chi(x) $ не является инт на $[a,b],$ а $f_{0}(x)$ - является
\begin{theorem}[Критерий Лебега] 
    Пусть $f(x)$ ограничена на $[a,b]$, а $R(f)-$ множество точек разрыва $f(x)$ на $[a,b],$ тогда $f(x)$ интегрируема по Риману на $[a,b] \Leftrightarrow R(f)$ имеет меру нуль, т.е $\forall \varepsilon>0 \exists \{\alpha_{i},\beta_{i}\}_{i=1}^{\infty}:R(f) \cup(\alpha_{i},\beta_{i}),$ приэтом $\underset{m}{sup} \sum_{i=1}^{m}(\beta_{i}-\alpha_{i})<\varepsilon$
\end{theorem}
\begin{proof}
    Без доказательства.
\end{proof}
\section{Линейные свойства определенного интеграла}

\begin{definition}
    Если $f(x)$ определена при $x=a$, то положим $\int _{a}^{a}f(x)dx\equiv 0$
\end{definition}
\begin{definition}
    Если $a<b,$ а еще $f(x)$ интегрируема на $[a,b]$, то положим $\int _{b}^{a}f(x)dx\equiv - \int _{a}^{b}f(x)dx$
\end{definition}
\begin{theorem}
    Если $f(x),g(x)$ интегрируемы на $[a,b]$, $f(x)\pm g(x)$ тоже интегрируема на $[a,b]$,причем $\int _{a}^{b}(f(x)\pm g(x)dx)= \int _{a}^{b}f(x)dx \pm \int _{a}^{b} g(x)dx$
\end{theorem}
\begin{proof}
    Если $a=b$, то доказывать нечего: $0=0\pm 0.$\\ 
    Если $a<b$, то: Берем $\forall T=\{a=x_{0}<x_{1}<x_{2}<\dots<x_{n}=b\};$ Берем $\forall \Xi=\{\xi_{k}\}_{k=1}^{n},$ тогда: 
    рассмотрим $\sigma_{T}(f\pm g,\Xi)=\sum_{k=1}^{n}(f(\xi_{k})\pm g(\xi_{k}))\Delta x_{k} =\sum_{k=1}^{n}f(\xi_{k})\Delta x_{k} \pm  \sum_{k=1}^{n} g(\xi_{k})\Delta x_{k}=\underset{\underset{\delta_{T}\to 0}{\to} I_{1}}{\sigma_{T}(f,\Xi)} \pm  \underset{\underset{\delta_{T}\to 0}{\to} I_{2}}{\sigma_{T}(g, \Xi)}\to I_{1}+I_{2},$ т.е $\int _{a}^{b}(f(x)\pm g(x))dx=\int _{a}^{b}f(x)dx \pm  \int _{a}^{b}g(x)dx$\\ 
    Если $a>b \implies \int _{b}^{a}(f(x)\pm g(x))dx=\int _{b}^{a}f(x)dx\pm \int _{b}^{a}g(x)dx= \int _{a}^{b}f(x)dx\pm \int _{a}^{b}g(x)dx$
\end{proof}
\begin{theorem}
    $f(x)$ интегрируема на $[a,b]\implies\forall c\in\mathbb{R}\quad c\in f(x)\text{ интегрируема на }[a,b]$,причем $\int _{a}^{b}cf(x)dx=c \int _{a}^{b}f(x)dx$
\end{theorem}
\begin{proof}
    Самостоятельно.
\end{proof}
\section{Интегрируемость произведения интегрируемых по Риману функций}
\begin{theorem}
    Если $f(x),g(x)$ интегрируемы на $[a,b]\implies f(x)g(x)$ тоже интегрируема на $[a,b]$
\end{theorem}
\begin{proof}
    пусть $a<b$
    $f(x),g(x)$ интегрируемы на $[a,b]\implies f(x),g(x) - \text{ ограничены на }[a,b]$, т.е $\exists M^{(f)}>0,M^{(g)}>0: \begin{aligned}
        |f(x)|\leqslant M^{(f)} \forall x\in[a,b] \\ 
        |g(x)|\leqslant M^{(g)} \forall x\in[a,b]
    \end{aligned}$. Берем $\forall T=\{a=x_{0}<x_{1}<\dots<x_{n}=b\};$ Введем $M_{k}^{f}=\underset{x_{k-1}\leqslant x\leqslant x_{k}}{supf(x)} \\ m_{k}^{(f)=inff(x)}$\\ 
    $M_{k}^{(g)}=supg(x) \\ m_{k}^{(g)}=infg(x)$\\ $M_{k}^{(fg)}=sup(f(x)g(x)) \\ m_{k}^{fg}=inf(f(x)g(x))$\\ 
    $\forall \varepsilon>0 \exists x'_{k},x_{k}''\in[x_{k-1},x_{k}]:\begin{aligned}M_{k}^{(fg)}<f(x'_{k})g(x'_{k})+\frac{\varepsilon}{2}\\ m_{k}^{(fg)}>f(x''_{k})g(x''_{k})-\frac{\varepsilon}{2}\end{aligned}
    \implies 0\leqslant M_{k}^{(fg)}-m_{k}^{(fg)}<f(x_{k}')g(x'_{k})-f(x_{k}'')g(x_{k}'')+\varepsilon =f(x'_{k})g(x_{k}')+f(x'_{k})g(x_{k}'')-f(x'_{k})g(x_{k}'')-f(x_{k}'')g(x''_{k})+\varepsilon=f(x'_{k})(g(x_{k}')-g(x_{k}''))+g(x_{k}'')(f(x_{k}')-f(x_{k}''))+\varepsilon\leqslant M^{f}((M_{k})^{g}-m_{k}^{(g)}) +M^{g}(M_{k}^{(f)}-m_{k}^{(f)})+\varepsilon\implies 0\leqslant M_{k}^{(fg)}-m_{k}^{(fg)}\leqslant M^{(f)}((M_{k})^{g}-m_{k}^{(g)})+M^{(g)}(M^{f}_{k}-m_{k}^{f})\implies 0\leqslant S_{T}(fg)-s_{T}(fg)\leqslant M^{(f)}(S_{T}(g)-s_{T}(g))+M^{(g)}(S_{T}(f)-s_{T}(f))\implies \lim\limits_{\delta_{T}\to 0} $
  
\end{proof}
\end{document}