\documentclass[../main.tex]{subfiles}
\begin{document}
\newpage
\lecture{2}{20.12}{}
\section{Способы вычисления неопределенных интегралов}
\subsection{Метод подстановки}
$\displaystyle\int f(u)du=F(u)+C\qquad u=\varphi(x) - \text{дифферен.}.f(\varphi(x))$ опр при$x \in-\text{промежуток}.$\\
Рассмотрим $F(\varphi(x)),x\in X.$ $(F(\varphi(x)))'_{x}=F'_{u}\bigg|_{u=\varphi(x)}(u)*\varphi'(x)=f(u)\bigg|_{u=\varphi(x)}*\varphi'(x)=f(\varphi(x))*\varphi'(x)$\\
$\displaystyle\int f(\varphi(x))dx=F(\varphi(x))+C$\\
\fbox{%
\begin{minipage}{\linewidth}
\[
\displaystyle\int f(\varphi(x))\underbrace{\varphi'(x)dx}_{du}=\int f(u)du\bigg|_{u=\varphi(x)}
\]
\[
\displaystyle\int f(x)dx\bigg|_{x=\psi(t)}=\int f(\psi(t))\psi'(t)dt\bigg|_{t=\psi^{-1}(x)}
\]
\end{minipage}
}

\subsubsection{Примеры.}

\noindent 1. 
\[
\int \frac{x \, dx}{x^{2} + a^{2}} = \frac{1}{2} \int \frac{du}{u} = \frac{1}{2} \ln{|u|} + C
\]
\[
x^{2} + a^{2} = u, \quad 2x \, dx = du
\]

\noindent 2. 
\[
\int \sin^{3}{x} \, dx = -\int \sin^{2}{x} (-\sin{x} \, dx) = -\int (1 - u^{2}) \, du = -u + \frac{u^{3}}{3} + C = -\cos{x} + \frac{\cos^{3}{x}}{3} + C
\]
\[
\cos{x} = u, \quad -\sin{x} \, dx = du
\]

\noindent 3. 
\[
\int \frac{dx}{\sqrt{x}(1 + \sqrt[3]{x})} = \int \frac{6t^{5} \, dt}{t^{3}(1 + t^{2})} = 6 \int \frac{t^{2} + 1 - 1}{1 + t^{2}} \, dt = 6(t - \arctan{t}) + C = 6(\sqrt[6]{x} - \arctan{\sqrt[6]{x}}) + C
\]
\[
x = t^{6} (\sqrt[6]{x} = t), \quad dx = 6t^{5} \, dt
\]

\noindent 4. 
\[
\int \cot{x} \, dx = \int \frac{\cos{x}}{\sin{x}} \, dx = \int \frac{du}{u} = \ln{|u|} + C = \ln{|\sin{x}|} + C
\]
\[
\sin{x} = u, \quad du = \cos{x} \, dx
\]

\subsection{Интегрирование по частям}
$d(uv)=udv+vdu,udv=d(uv)-vdu,\quad \displaystyle\int udv=\int d(uv)-\int vdu;\quad \int d(uv)=uv+C$\\
$$\fbox{\(\displaystyle \int  udv=uv-\int vdu\)}$$

\subsubsection{Примеры}

\noindent 1. 
\[
\int \underbrace{\ln{x}}_{u} \underbrace{dx}_{dv} = x \ln{x} - \int x \frac{dx}{x} = x \ln{x} - \int dx = x \ln{x} - x + C
\]
\[
u = \ln{x}, \quad dv = dx, \quad du = \frac{dx}{x}, \quad v = x
\]

\noindent 2. 
\[
I = \int e^{x} \underbrace{\cos{x} \, dx}_{d(\sin{x})} = \int e^{x} d(\sin{x}) = e^{x} \sin{x} - \int \sin{x} e^{x} \, dx = e^{x} \sin{x} + \int e^{x} d(\cos{x}) = e^{x} \sin{x} + e^{x} \cos{x} - \underbrace{\int \cos{x} e^{x} \, dx}_{I}
\]
\[
I = \int e^{x} \cos{x} \, dx = \frac{1}{2} e^{x} (\sin{x} + \cos{x}) + C
\]

\noindent 3. 
\[
I = \int \underbrace{\sqrt{x^{2} + a^{2}}}_{u} \underbrace{dx}_{dv} = x \sqrt{x^{2} + a^{2}} - \int x \frac{2x \, dx}{2 \sqrt{x^{2} + a^{2}}} = x \sqrt{x^{2} + a^{2}} - \int \frac{x^{2} + a^{2} - a^{2}}{\sqrt{x^{2} + a^{2}}} \, dx\] \[ = x \sqrt{x^{2} + a^{2}} - I + a^{2} \ln{(x + \sqrt{x^{2} + a^{2}})}
\]
\[
I = \frac{x}{2} \sqrt{x^{2} + a^{2}} + \frac{a^{2}}{2} \ln{(x + \sqrt{x^{2} + a^{2}})} + C
\]

\noindent 4. 
\[
J_{n} = \int \frac{dx}{(x^{2} + a^{2})^{n}} \qquad a > 0, \quad n \in \mathbb{N}
\]
\[
J_{n} = \int \frac{1}{(x^{2} + a^{2})^{n}} \, dx = \frac{x}{(x^{2} + a^{2})^{n}} - \int x \frac{(-n)}{(x^{2} + a^{2})^{n+1}} 2x \, dx = \]
\[ \frac{x}{(x^{2} + a^{2})^{n}} + 2n \int \frac{x^{2} + a^{2} - a^{2}}{(x^{2} + a^{2})^{n+1}} \, dx = \frac{x}{(x^{2} + a^{2})^{n}} + 2n (J_{n} - a^{2} J_{n+1})
\]
\[
J_{n+1} = \frac{1}{2n a^{2}} \left[ \frac{x}{(x^{2} + a^{2})^{n}} + (2n - 1) J_{n} \right] \qquad n = 1, 2, 3, \dots
\]
\[
J_{1} = \int \frac{1}{x^{2} + a^{2}} \, dx = \frac{1}{a} \arctan{\frac{x}{a}} + C
\]
\[
J_{2} = \int \frac{dx}{(x^{2} + a^{2})^{2}} = \frac{1}{2a^{2}} \left[ \frac{x}{x^{2} + a^{2}} + \frac{1}{a} \arctan{\frac{x}{a}} \right] + C, \dots
\]

\noindent В качестве упражнения найти рекуррентную формулу для 
\[
\int \frac{dx}{(ax^{2} + bx + c)^{n}}
\]

\noindent
\[
\int \frac{P_{m}(x)}{Q_{n}(x)} \, dx \qquad \deg{P_{m}(x)} = m, \quad \deg{Q_{n}(x)} = n
\]
\[
m > n \quad P_{m}(x) = R_{m-n}(x) Q_{n}(x) + T_{k}(x), \quad k < n \quad \frac{P_{m}(x)}{Q_{n}(x)} = R_{m-n}(x) + \frac{T_{k}(x)}{Q_{n}(x)}, \quad k < n
\]
\[
Q_{n}(x) = a_{0} x^{n} + \dots + a_{n} = a_{0} (x - x_{1})^{\alpha_{1}} \cdots (x - x_{l})^{\alpha_{l}} (x^{2} + p_{1} x + q_{1})^{\beta_{1}} \cdots (x^{2} + p_{r} x + q_{r})^{\beta_{r}}
\]
\[
\frac{T_{k}(x)}{Q_{n}(x)} = \frac{A_{1}}{x - x_{1}} + \dots + \frac{A_{1 \alpha_{1}}}{(x - x_{1})^{\alpha_{1}}} + \dots + \frac{A_{\alpha_{1} 1}}{x - x_{l}} + \dots + \frac{A_{\alpha_{1} \alpha_{l}}}{(x - x_{l})^{\alpha_{l}}} + \frac{B_{11} x + C_{11}}{x^{2} + p_{1} x + q_{1}} + \dots + \frac{B_{1 \beta_{1}} x + C_{1 \beta_{1}}}{(x^{2} + p_{1} x + q_{1})^{\beta_{1}}} + \dots
\]
\end{document}