\documentclass[../main.tex]{subfiles}
\begin{document}


\textbf{Определение.} $\forall n\in\mathbb{N}$ поставлено в соответствие $x_{n}\in\mathbb{R},$ то $\{x_{n}\}^{\infty}_{n=1}={x_{1},x_{2},\dots,x_{n}}$ - называется числовой последовательностью.\\
$\{x_{n}\}^{\infty}_{n=0}={x_{0},x_{1},\dots,x_{n}}$\\
$x_{n}=\frac{1}{n};\{\frac{1}{n}\}^{\infty}_{n=1}=\{1,\frac{1}{2},\dots,\frac{1}{n},\dots \}$\\
$x_{n}=(-1)^{n}\quad \{(-1)^{n}\}^{\infty}_{n=0}=\{1,-1,1,-1,\dots\}$
\textbf{Определение.} $\{x_{n}\},\{y_{n}\}$ - последовательности $\implies \{x_{n}\pm y_{n}\}$ - сумма(разность) последовательностей. \\
$\{x_{n}*y_{n}\}$ - произведение.$\qquad y_{n}\neq 0\; \forall n\geqslant n_{0} \implies \{\frac{x_{n}}{y_{n}}\}$ - частное. \\
\textbf{Определение.} $\{x_{n}\}$ называется ограниченной сверху, если $\exists M\in\mathbb{R}: \forall n\;x_{n}\leqslant M\;$\\
\textbf{Определение.} $\{x_{n}\}$ называется ограниченной снизу, если $\exists m\in\R : \forall n \; x_{n}\geqslant m$.\\
\textbf{Определение.} $\{x_{n}\}$ называется ограниченной, если ограничена и сверху и снизу.\\ 
$\{x_{n}\}$ - ограничена $\Leftrightarrow\exists A\geqslant 0 : \forall n |x_{n}| \leqslant A$\\
$\rightarrow \qquad \exists M,m : \forall n\; m\leqslant x_{n} \leqslant M \qquad A=max(|m|,|M|)\implies -A \leqslant x_{n} \leqslant A$
\\$\leftarrow \qquad \exists A\geqslant 0 : \forall n \; |x_{n}|\leqslant A \implies -A \leqslant x_{n} \leqslant A\qquad A=M \qquad m\leqslant x_{n}\leqslant M\\$

$\{x_{n}\} - $не ограничена $\Leftrightarrow \forall A\geqslant 0 \implies \exists n: |x_{n}| \geqslant A$\\
\begin{math}
    x\in\mathbb{R} [x]=F(x) - \text{целая часть }  x \qquad [x] \text{ - наибольшее целое } \geqslant x
    [1] = 1 \quad [\pi] = 3 \quad [-\pi] = -4 \quad x-1 < [x] \leqslant x \qquad [x] \leqslant x <[x] +1
\\x_{n} =n=\{1,2,\dots,n,\dots\} - \text{ неограниченная } \forall A \geqslant 0 \exists n= [A] +1 > A(x_{n}=n>A)
\\x_{n}=\frac{1}{n} -\text{ ограниченная  }\; 0<x_{n}=\frac{1}{n} \leqslant 1 \forall n \\
x_{n} = (-1)^{n} - \text{ ограниченная } \quad |x_{n}|=1(|x_{n}\leqslant 1)\\
\{n^{-1^{n}}\}_{n=1}^{\infty} = \{\frac{1}{1},2,\frac{1}{3},\dots,\frac{1}{2n-1},2n,\dots\} -\text{ неограниченная }  
\forall A\geqslant 0 \exists n = \underbrace{(2[A]+1)>2A}_{\text{четн}}\geqslant A  \\
\end{math}

\subsection{Предел последовательности}
\textbf{Определение.} a - предел последовательности $\{x_{n}\}$, если $\forall \epsilon >0 \exists N : \forall n>N \quad |x_{n}-a|<\epsilon\qquad  a=\lim_{x\to \infty}\{x_{n}\}$\\\\
Если последовательность имеет предел, то она называется сходящейся, а если не имеет - расходящейся.\\
$\{x_{n}\}$ - сходится, если $\exists a:\forall \epsilon >0 \exists N:\forall n> N\quad |x_{n}-a|<\epsilon$\\
$\{x_{n}\}$ - расходится, если $\forall a \exists \epsilon > 0 \forall N \exists n > N \quad |x_{n}-a| \geqslant \epsilon$

\end{document}