\documentclass[../main.tex]{subfiles}
\begin{document}

\lecture{1}{02.09}{}
\section{Множества и действия над ними.}
Множество, элемент множества - неопределимые понятия.
$a\in A;$ a - элемент; A - множество. ;$A\equiv\{a,...,\}$\\
Некоторые обозначения:\\
$\forall$ - для каждого, для любого, для всякого;\\
$\exists$ - существуе, найдется;\\
$\in$ - принадлежит, входит;\\
$\implies$ - следует, влечёт;\\
$\Leftrightarrow$ - тогда и только тогда;\\
$:$ - так, что\\
\textbf{Определение.} A - подмножество множества B($A\in B$), если $\forall a \in A \implies a\in B | A\subset B$; В частности B=A\\
Определение A, B называются равными, если состоят из одних и тех же элементов A=B; В частности $ A\subset B, B\subset A \Leftrightarrow A=B$\\
$\varnothing \subset A\forall A$\\
\textbf{Определение.} Пересечением A и B называется: $A \cap B \equiv $\{$a:a\in A$ и $a\in B$\}\\
\textbf{Определение.} Объединением A и B называется: $A \cup B \equiv $\{$a:a\in A$ или $a\in B$\}\\
\textbf{Определение.} $\cup A_\alpha \equiv\{a: \exists\alpha, a \in A_\alpha\}$\\
\textbf{Определение.} $\cap A_\alpha \equiv\{a:\forall\alpha,a\in A_\alpha\}$\\
\textbf{Определение.} Разностью A и B называется $A\setminus B\equiv$\{$a:a \in A$ и $a\notin B$\}\\
\textbf{Определение.} Симметричной разностью A и B:$A\Delta B\equiv(A\setminus B)\cap (B\setminus A)$\\
\textbf{Определение.} Если $\forall a \in A$ по некотормоу закону(правилу) поставлен в соответствие единственный элемент $b\in B$ и 
$\forall b \in B$ оказался поставлен в соответствие единственный $a\in A$, то множества A и B называются эквивалентыми(равномощными),
а соответствие взаимооднозначным. A ~ B\\

\section{Числа и числовые множества}
\subsection{Натуральные числа $\mathbb{N}$}
\textbf{Определение.} A - конечное, если состоит из конечного числа элементов, т.е $\exists n\in \mathbb{N}: A~\{1,2,...,n\}$\\
\textbf{Определение.} A - счетное, если $A~\mathbb{N}\equiv\{1,2,3,..,n,..\}$\\
\subsection{Целые числа $\mathbb{Z}$}
$\mathbb{Z}\equiv\{0,\pm 1, \pm 2, ...\}$\\
\begin{math}
    |a| = 
    \begin{cases}
        a, a\geqslant 0\\
        -a, a < 0
    \end{cases}
\end{math}
\begin{math}
    a*b = 
    \begin{cases}
        0,&\text{если $a=0$ или $b=0$}\\
        |a|*|b|&\text{если a,b - одного знака}\\
        -|a|*|b|&\text{если a,b - разных знаков}
    \end{cases}
\end{math}
$\mathbb{N} \subset \mathbb{Z}$; $\mathbb{Z}\setminus\mathbb{N}\neq\varnothing$\\
0,1,-1,2,-2,...,n,-n,...\\
1,2,3,4,5,...,2n,2n+1\\
Если множество элементов эквивалентно своей части, то оно обязательно бесконечное.
\subsection{Рациональные числа $\mathbb{Q}$}
$\mathbb{Q} \equiv\{\frac{p}{q} : p\in \mathbb{Z}, q\in \mathbb{N}\}$\\
$\mathbb{Z} ~\mathbb{Z}_1\equiv\{\frac{p}{1};p\in\mathbb{Z}\}\subset\mathbb{Q}$\\
$\frac{p}{q} \in\mathbb{Q}   r=|p|+|q|$ - "вес" числа $\frac{p}{q}$\\
    $r=1:(\frac{0}{1})_1
    r=2:\frac{0}{2} (\frac{1}{1})_2 (\frac{-1}{1})_3\\
    r=3:\frac{0}{3} (\frac{1}{2})_4 (\frac{-1}{2})_5 (\frac{2}{1})_6 (\frac{-2}{1})_7\\
    r=4:\frac{0}{4} (\frac{1}{3})_8 (\frac-1/3)_9 \frac{2}{2} \frac{-2}{2} (\frac{3}{1})_10 (\frac{-3}{1})_11$\\
\begin{math}
    1. a>b,b>c \implies a>c(a=b,b=c\implies a=c)\\
    2. a+b = b+a\\
    3.(a+b)+c=a+(b+c)\\
    4.\exists 0 : a-0=a\forall a\\
    5.\forall a\exists a^{\prime}:a+a'=0\\
    6.ab=bab\\
    7.|ab|c=a|bc|\\
    8.\exists 1\neq0: a*1 = a\forall a\\
    9.\forall a \neq \exists a'' :a*a''=1\\
    10. (a+b)c=ac+bc\\
    11.a>b\implies\forall c a+c>b+c\\
    12.a>b \implies\forall c>0 ac>bc\\
    13.\forall a  \exists n\in  \mathbb{N} : \underbrace{1+1+...+1}_n >a\\\\
    a>b,c>d \implies a+c>b+d\\
    a>b\implies a+c> b+c\\
    c>d \implies b+c> b+d \implies a+c>b+d\\
\end{math}
Предположим, что $\sqrt{2} = \frac{p}{q}$ - несократимая дробь.\\
$\frac{p^2}{q^2}=2 \implies p^2=2q^2\implies p^2\vdots 2\implies 4k^2=2q^2\implies q^2=2k^2\implies q^2\vdots2
\implies q^2\vdots 2\implies q\vdots 2 \implies \frac{p}{q}$ - сократимая - противоречие 
$\implies \sqrt{2}\neq\mathbb{Q}$\\
\end{document}