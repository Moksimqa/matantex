\documentclass[../main.tex]{subfiles}
\begin{document}

\lecture{1}{13.12}{}
\section{Первообразная и неопределенный интеграл}

\begin{definition}
     $X$ - промежуток $f(x)$ определена $x\in X. F(x), x\in X,$ называется первообразной к $f(x)$, если $\forall x \in X\; \exists F'(x)=f(x)$
\end{definition}
\begin{definition}
    Множество всех первообразных к $f(x)$ на $X$ называется неопределенным интегралом.(об.$\displaystyle\int{f(x)dx}, f(x)$ - подынтегральная функция,$f(x)dx$ - подынтегральное выражение)\\
\end{definition}

$F(x) - $ первообразная к $f(x)\implies F(x)+C$ - тоже первообразная.\\
$\varPhi(x)$ - первообразная к $f(x)\qquad F'(x)=f(x)=\varPhi'(x)$\\
Рассмотрим $(\varPhi(x)-F(x))'=\varPhi'(x)-F'(x)=f(x)-f(x)=0\;\forall x \in X$\\ 
$\varPhi(x)-F(x)=C=const\qquad \varPhi(x)=F(x)+C$\\
$\displaystyle\int f(x)dx=F(x)+C,\quad d(\int (f(x)dx))=f(x)dx;\;\int dF(x)=F(x)+C$\\
$dF(x)=F'(x)dx=f(x)dx\qquad d(\displaystyle\int f(x)dx)=d(F(x)+C)=f(x)dx$

\begin{theorem}
    $f(x),g(x)$ имеют первообразные $\implies f(x)\pm g(x)$ тоже имеют первообразные, причем $\displaystyle\int(f(x)\pm g(x))dx=\int f(x)dx \pm \int g(x)dx.$
\end{theorem}
\begin{proof}
    $\displaystyle\int f(x)dx=F(x)+C_{1},\int g(x)dx=G(x)+C_{2}.$ Рассмотрим $H_{\pm}(x)=F(x)\pm G(x), H'_{\pm}(x)=F'(x)\pm G'(x)=f(x)\pm g(x)$\\ 
$\displaystyle\int(f(x)\pm g(x))dx=F(x)\pm G(x)+C=\int f(x)dx+\int g(x)dx$ 
\end{proof}
\begin{theorem}
    $f(x)$ имеет первообразную $\implies \forall k, kf(x)$ тоже имеет первообразную, а если $k \neq 0,$ то $\int kf(x)dx=k\int f(x)dx$
\end{theorem}
\begin{proof}
$\displaystyle\int f(x)dx=F(x)+C\qquad (kF(x))'=kF'(x)=kf(x)\qquad\int kf(x)dx=kF(x)+C_{1}\qquad k\int f(x)dx=k(F(x)+C)=kF(x)+kC,\quad \int kf(x)dx=kF(x)=C_{1},\quad k\int f(x)dx=k(F(x)+C)=kF(x)+kC,$ если $k\neq 0,$
то $\displaystyle\int kf(x)dx=k\int f(x)dx$
\end{proof}


\subsection{Таблица интегралов.} 
\noindent 1. $\displaystyle \int x^{\alpha}dx=\frac{x^{\alpha+1}}{\alpha+1}+C,\alpha\neq -1(\alpha=0\implies x^{\alpha}=1)$\\
2. $\displaystyle \int \frac{dx}{x}=ln|x|+C$\\
3. $\displaystyle \int a^{x}dx=\frac{a^{x}}{lna}+C$\\
4. $\displaystyle \int  sinxdx=-cosx+C$\\
5. $\displaystyle\int cosxdx=sinx+C$\\
6. $\displaystyle \int \frac{dx}{cos^{2}x}=tgx+C$\\
7. $\displaystyle \int \frac{dx}{sin^2x}=-ctgx+C$\\
8. $\displaystyle \int shxdx=chx+C$\\ 
9. $\displaystyle \int chxdx=shx+C$\\
10. $\displaystyle \int  \frac{dx}{ch^{2}x}=thx+C$\\
11. $\displaystyle \int  \frac{dx}{sh^{2}x}=-cthx+C$\\
12. $\displaystyle \int \frac{dx}{x^{2}+a^{2}}=\frac{1}{a} \arctan{\frac{x}{a}}+C=-\frac{1}{a}\arctan{\frac{x}{a}}+C,a>0$\\
13. $\displaystyle \int  \frac{dx}{x^2-a^2}=\frac{1}{2a}\ln{\left|\frac{x-a}{x+a}\right|}+C$\\
14. $\displaystyle\int \frac{dx}{\sqrt{a^{2}-x^{2}}}=\arcsin{\frac{x}{a}}+C=-\arccos{\frac{x}{a}}+\tilde{C},a>0$\\
15.$\displaystyle \int \frac{dx}{\sqrt{x^{2}\pm a^{2}}}=\ln{|x+\sqrt{x^{2}\pm a^{2}}|}+C,a>0$
\end{document}