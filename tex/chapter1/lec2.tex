\documentclass[../main.tex]{subfiles}
\begin{document}

\lecture{2}{}{}
\subsection{Действительные (вещественные) числа $\mathbb{R}$}
$0\Leftrightarrow 0,00...0...=+0,00...0...\\$
M - справа от 0, отложим OF $a_0$ раз $\implies 0<NM\leqslant 1  0<PM\leqslant \frac{1}{10}\\
a_0,a_0a_1=a_0+\frac{a_1}{10};\\
a_0,a_1,a_2=a_0+\frac{a_1}{10^1}+\frac{a_2}{10^2};\\
a_0,a_1,...,a_n=a_0+\frac{a_1}{10^1}+\frac{a_2}{10^2}+...+\frac{a_n}{10^n}; ... ; a_0a_1...a_n =+a_0a_1...$\\

Если M - слева от 0 $\implies $ раскладываем 1-ый отрезок $a_0$ раз, затем
$\frac{1}{10}$ отрезок и т.д $\implies -a_0,a_1,a_2,...,a_n...$\\
Если M=E$\implies a_0=0;a_1=9=a_2=...=a_n=...\implies E=0,99...9...\\$
Каждой точке оси поставили в соответствие десятичную дробь; $x=0,99...; 10x=9,99...=9+x x=1$\\
\textbf{Определение.} Действительным(вещественным) числом называется бесконечная десятичная дробь, поставленная в соответствие точке вещественной оси.\\
\textbf{Определение.} Вещественные числа $a=\pm a_0,a_1,...a_n...,$ называется неотрицательным, если оно со знаком +
(который может быть опущен) и отрицательным, если оно со знаком -\\
%\textbf{Определение.} Абсолютной величиной от $a=\pm a_0,a_1,...,a_n,...$ называется число $|a|=a_0,a_1,...,a_n,...$\\
%\textbf{Определение.} Два вщеественных числа $a=a_0,a_1,...,a_n,...$ и $b=b_0,b_1,...,b_n,...$ называются равными(a=b), если у них одинаковые
%знаки и $a_k=b_k(k=0,1,2,...)$\\
$a=b;b=c$(у них одинаковые знаки)$\implies a_k=b_k; b_k=c_k \implies a_k=c_k \forall k\implies a=c$\\
\textbf{Определение.} Неотрицательно вещественное a называется положительным, если $a\neq 0(0=0,000,...,0,...)$\\
\textbf{Определение.} $a=\pm a_0,a_1,...,a_n,..., b=\pm b_0,b_1,...b_n,... a\neq b$\\
$a,b$ - неотрицательные $\exists m\geqslant:a_0=b_0;a_1=b_1,...a_{m-1}=b_{m-1}; a_m\neq b_m\\
a_n>b_n$ тогда $a>b$\\
$a_n<b_n$ тогда $a<b$\\
a - неотрциательное, b - отрицательное, тогда $a>b$\\
a,b - отрицательные, тогда, если $|a|>|b>\implies b>a\\
|a|<|b|\implies a>b$\\
a - неотрицальное $\implies a\geqslant 0$\\
b - положительное $\implies b>0$\\
c - отрицательное $\implies c<0$\newpage
$a>b,b>c \implies a>c(?)$\\
Пусть a,b,c - неотрицальные \\$a=a_0,a_1,...,a_n,...;\;
b=b_0,b_1,...,b_n,...;\;
c=c_0,c_1,...,c_n,...;\;\\
\exists p,q:a_0=b_0...a_{p-1} = b_{p-1}; a_p >b_p \\p,q\geqslant 0 b_0=c_0...b_{q-1}=c_{q-1};
b_q>c_q \\\exists r=min(p,q):a_0=b_0=c_0... a_{r-1}=b_{r-1}=c_{r-1} a_r\geqslant b_r\geqslant c_r\implies a_r>c_r\implies a>c$\\

\section{Ограниченные и неограниченные множества}
\textbf{Определение.} $X \subset \mathbb{R}$ называется ограниченным сверху, если $\exists M \in \mathbb{R}:\forall x\in X \implies x\leqslant M.$ M - верхняя грань.\\
\textbf{Определение.} $X \subset \mathbb{R}$ называется ограниченным снизу, если $\exists m \in \mathbb{R}:\forall x\in X \implies x\geqslant m.$ m - нижняя грань.\\
\textbf{Определение.} Множество называется ограниченным, если оно ограничено и сверху, и снизу.\\\\

$X$ не ограничено сверху, если $\forall M \in \mathbb{R}\implies \exists x\in X: x > M$ $M=supX$\\
\textbf{Определение.} $M\in\mathbb{R}$ называется точной верхней гранью $X\subset \mathbb{R}$, если \\
$1) \forall x \in X\implies x\leqslant M\\
2) \forall M_1 < M \exists x \in X : x> M_1$\\\\

$M\neq supX,$ если\\ 
или $\exists x\in X: x> M$\\
или $\exists M_1 <M:\forall x \in X\implies x\leqslant M_1$\\
\textbf{\textbf{Определение.}} Пусть $m\in \mathbb{R}$ m - точная нижняя грань $X \subset \mathbb{R}$($m=infX$),если\\
$1)\forall x\in X \implies x\geqslant m\\
2)\forall m_1 > M\exists x \in X: x< m_1$\\
$X$ - неограниченное сверху: $supX = +\inf$\\
$X$ - неограниченное снизу: $infX =-\inf$\\ 
\textbf{Теорема 1.} $\forall X \subset \mathbb{R} : X \neq \varnothing, X$ - ограниченно сверху
$\implies \exists M = supX(M\in\mathbb{R})$\\
\textbf{Теорема 2.}$\forall X \subset \mathbb{R}: X\neq \varnothing, X$ - ограничено снизу
$\implies \exists m = infX(m\in\mathbb{R})$\newpage
\end{document}