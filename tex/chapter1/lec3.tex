\documentclass[../main.tex]{subfiles}
\begin{document}
\lecture{3}{13.09}{}
%\date{13.09.2024}\\ 
$M=supX$, если\\
1)$\forall x\in X \implies x\leqslant M_1\\
2)\forall M_1< M\exists y\in X: y> M_1$\\
\textbf{Теорема 1.}$\forall\mathbb{X} \subset\mathbb{R}:\mathbb{X}\neq\varnothing , \mathbb{X}$ - ограничено сверху $\implies \exists M=sup\mathbb{X}$\\
$ x\neq\varnothing,\mathbb{X}$ - ограничено сверху $1) \exists x\in\mathbb{X} : x$  - неотрицательно, $x$ - ограничено сверху,
\\т.е $\exists\tilde{M} :\forall x\in\mathbb{X}\implies x\leqslant\tilde{M}$\\
$x=x_0,x_1,...,x_n,... \in \mathbb{X} x\leqslant\tilde{M} \{x_0\} \implies x_0\leqslant\tilde{M}$\\
$\bar{x} - max\{x_0\}$\\
$x=\bar{x_0},x_1,x_2,...,x_n<... \in\mathbb{X} \{x_1\}  \bar{x_1}=max\{x_1\},...\\
x=\bar{x_0},\bar{x_1},...,\bar{x_{k-1}},x_k,x_{k+1} \in\mathbb{X}\\$
$\{x_k\},\bar{x_k}=max\{x_k\},...  M=\bar{x_0},\bar{x_1},...,\bar{x_n},...\\$
$M=sup\mathbb{X}(?) M\geqslant 0$\\
$\forall x\in\mathbb{X}  \qquad x<0 \implies M>x ; x\geqslant 0,$ т.е $x=+x_0,x_1,...,x_n,... \in\mathbb{X}$\\
$x_0\leqslant\bar{x_0}\qquad x=\bar{x_0} \implies x_1\leqslant\bar{x_1}...x_{k-1}\leqslant\bar{x_{k-1}}\\$
$2)\forall M_1<M_1$, если $M_1<0$ $\exists y=x_0,x_1,...\geqslant 0 , y\in\mathbb{X}\implies y>M_1;$\\
если $M_1\geqslant 0,$но $M_1<M\quad M_1=\tilde{x_0},\tilde{x_1},...,\tilde{x_n},...\\
M_1<M=\bar{x_0},bar{x_1},...,\bar{x_n},...\quad \tilde{x_0}=\bar{x_0},...,\tilde{x_{k-1}}==\bar{x_{k-1}},$но $\tilde{x_k}<\bar{x_k}$\\
\\
$x=\bar{x_0},\bar{x_1},...,\bar{x_k},\bar{x_{k+1}},...\in\mathbb{X}\implies\exists y\in\mathbb{X} : y=\bar{x_0},\bar{x_1},...,\bar{x_k},x_{k+1}\in\mathbb{X} y>M1$\\
$M=sup\mathbb{X}$\\
$2)\forall x\in\mathbb{X}\implies x<0 \quad x=-x_0,x_1,...,x_n,...$. Проделаем то же самое, что делали для построения числа M, но будем брать 
min вместо max, и получим бесконечную десятичную дробь. $M=-\bar{x_0},\bar{x_1},...,\bar{x_n},...\\
M=-0,000...\implies M=+0,000....\\
M=-\bar{x_0},\bar{x_1},\dots,\bar{x_{k-1}},000...(\bar{x_{k-1}\neq 0}) \implies M = -\bar{x_0},x_1,...,\bar{x_{k-2}}(\bar{x_{k-1}}-1)*99$\\
$[a,b]\equiv\{x\in\mathbb{R}:a\leqslant x\leqslant b\}\qquad a,b\in\mathbb{R}(a<\leqslant b)$\\
$[a,b)\equiv\{x\in\mathbb{R}:a\leqslant x < b\}\qquad$или $ a\in\mathbb{R}$,или $b\in\mathbb{R},(a<b)$, или $b=+\infty$\\
$(a,b]\equiv\{x\in\mathbb{R}:a < x\leqslant b\}\qquad a\in\mathbb{R}(a<b),$или $a=-\infty,b\in\mathbb{R}$\\
$(a,b)\equiv\{x\in\mathbb{R}:a < x < b\}\qquad a,b\in\mathbb{R}(a<b)$и/или $a=-\infty,b=+\infty$\\
\\
(0,1) - несчетное множество.(?)\\
Предположим, что (0,1) - счетное.
\\$x^{(1)}=0,{x_1}^{(1)}{x_2}^{(1)}...{x_n}^{(1)}...$\\
$x^{(2)}=0,{x_1}^{(2)}{x_2}^{(2)}...{x_n}^{(2)}...$\\
.......................................\\
$x^{(n)}=0,{x_1}^{(n)}{x_2}^{(n)}...{x_n}^{(n)}...$\\
.......................................\\ 
$x=0,x_1x_2...x_n... :\; x_1\neq 0,x_1\neq 9,x_1\neq{x_1}^{(1)}\\
x_2\neq 0,x_2\neq 9,x_2\neq{x_2}^{(2)}\\
.......................................\\
x_n\neq 0,x_n\neq 9,x_n\neq{x_n}^{(n)}\\
.......................................\\
x\neq x^{(1)} \quad x_2\neq x^{(2)},...,x_n\neq x^{(n)}...\implies (0,1)$ - несчетно\\
$\forall x\in\mathbb{R}\forall n\in\mathbb{N} \exists x^{-}_{n},x^{+}_{n}\in\mathbb{Q} : x^{-}_{n}\leqslant x\leqslant x^{+}_{n}, x^{+}_{n}-x^{-}_{n}=\frac{1}{10^{n}}$\\
\textbf{Определение.} Суммой $x,y\in\mathbb{R}$ называется $x+y \equiv sup\{x^{-}_{n}+y^{-}_{n}\}\qquad(inf(x^{+}_{n}+y^{+}_{n}))$\\
\textbf{Определение.} Произведениями $x,y>0(x,y\in\mathbb{R})$ называется $x*y\equiv sum \{x^{-}_{n}*y^{-}_{n}\} (inf \{x^{+}_{n}*y^{+}_{n}\})$

\textbf{Определение.} Произведением $x,y\in\mathbb{R}$ называется \\
\begin{equation}
    x*y=
    \begin{cases}
        0, \text{если x=0 или y=0}\\
        |x|*|y|,\text{если x,y одного знака}\\
        -|x|*|y|,\text{если x,y разных знаков}
    \end{cases}
\end{equation}
$|x*y|=|x|*|y|\quad -|x|\leqslant x\leqslant|x| \quad -|y|\leqslant y\leqslant |y|\\
|x|+|y|\geqslant x+y \qquad |x|+|y|\geqslant -(x+y)\quad \implies |x|+|y|\geqslant|x+y| \qquad |x+y|\leqslant|x|+|y|$\\
$-|x|-|y|\leqslant x+y\leqslant|x|+|y| \qquad -(|x|+|y|) \leqslant x+y \leqslant |x|+|y| \qquad |x| = |y+(x-y)|\leqslant |y| +|x-y| \implies |x-y|\geqslant |x|-|y| \qquad
|x-y|=|y-x|\geqslant |y|-|x| \implies |x-y|\geq ||x|-|y||$
\subsection{Комплексные числа $\mathbb{C}$}
\textbf{Определение.} $z=(x,y)=z+iy, x,y\in\mathbb{R} x=Rez, y=Imz$\\
\textbf{Определение.} $z_{1}=x_{1}+iy_{1}, z_{2}=x_{2}+iy_{2} \quad z_{1}=z_{2}, \text{ если } x_{1}=x_{2} \text{ и }y_{1}=y_{2} $\\
\textbf{Определение.} $z_{1}=x_{1}+iy_{1}, z_{2}=x_{2}+iy_{2};\; z_{1}\pm z_{2}=(x_{1}\pm x_{2})+i(y_{1}\pm y_{2})$\\
$z_{1}*z_{2}=x_{1}*x_{2}-y_{1}y_{2}+i(x_{1}y_{2}+x_{2}y_{1})\\$
$z_{2}\neq 0+0i,\text{ то } \frac{z_{1}}{z_{2}}=\frac{x_{1}x_{2}+y_{1}y_{2}}{x_{1}^{2}+y_{2}^{2}}+\frac{x_{2}y_{1}-x_{1}y_{2}}{x_{2}^{2}+y_{1}^{2}}$\\
\end{document}