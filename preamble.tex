% ----------------------------------------------------
%    Основные настройки: кодировка, язык и шрифты
% ----------------------------------------------------
\usepackage[utf8]{inputenc}      % Поддержка UTF-8 для ввода
\usepackage[T2A]{fontenc}        % Кириллические шрифты (кодировка T2A)
\usepackage{textcomp}            % Дополнительные текстовые символы
\usepackage[russian]{babel}      % Русификация: переносы, лексика и т.д.
% ----------------------------------------------------
%    Пакеты для форматирования ссылок и гиперссылок
% ----------------------------------------------------
\usepackage{url}                 % Поддержка гиперссылок с форматированием URL
\usepackage{hyperref}            % Гиперссылки в PDF-документе
    \hypersetup{
        colorlinks=true,
        linkcolor=blue,
        filecolor=magenta,      
        urlcolor=cyan,
        pdftitle={Математический анализ},
        pdfpagemode=FullScreen,
    } 
    
% ----------------------------------------------------
%    Работа с графикой и рисунками
% ----------------------------------------------------
\usepackage{graphicx}            % Вставка графики
\usepackage{float}               % Улучшенное позиционирование плавающих объектов
\usepackage{tikz}                % Создание графики средствами TikZ
%\usepackage{pgfplots}           % Построение графиков (не используется)

% ----------------------------------------------------
%    Таблицы и списки
% ----------------------------------------------------
\usepackage{booktabs}            % Красивые таблицы с горизонтальными линиями
\usepackage{enumitem}            % Гибкая настройка списков enumerate/itemize
\usepackage{multirow}            % Объединение строк в таблицах
% ----------------------------------------------------
%    Дополнительные пакеты
% ----------------------------------------------------
\usepackage{epigraph}            % Оформление эпиграфов
\usepackage{subfiles}            % Подключение дочерних TeX-файлов
\usepackage{titling}             % Гибкость при оформлении заголовка документа
\usepackage{fancybox}            % Различные рамки и блоки
\usepackage{indentfirst}         % Отступ у первого абзаца раздела
% ----------------------------------------------------
%    Геометрия и макет страницы
% ----------------------------------------------------
\usepackage[left=2cm,right=2cm,top=2cm,bottom=3cm,bindingoffset=0cm]{geometry} % Настройка полей страницы
% --- Работа с PDF ---
\pdfminorversion=7                         % Минимальная версия PDF (совместимость)
\usetikzlibrary{decorations.pathreplacing} % Расширения TikZ для декораций

% --- Абзацы без отступа, с интервалами ---
\usepackage{parskip}           % Убирает отступы, добавляет вертикальные интервалы

% --- Пустые страницы без номеров ---
\usepackage{emptypage}         % Не нумерует пустые страницы

% --- Подписи к подрисункам и мультиколонки ---
\usepackage{subcaption}        % Подрисунки в рамках одной фигуры
\usepackage{multicol}          % Несколько колонок

% --- Цвета ---
\usepackage{xcolor}            % Работа с цветом

% ----------------------------------------------------
%    Математические пакеты
% ----------------------------------------------------
\usepackage{amsmath, amsfonts, mathtools, amsthm, amssymb} % Базовые математические пакеты
\usepackage{mathrsfs}          % Красивые каллиграфические символы
\usepackage{cancel}            % Зачёркивание выражений
\usepackage{bm}                % Жирные символы в формулах


% ----------------------------------------------------
%    Собственные обозначения и команды
% ----------------------------------------------------
\newcommand\N{\ensuremath{\mathbb{N}}} % Натуральные числа
\newcommand\R{\ensuremath{\mathbb{R}}} % Вещественные числа
\newcommand\Z{\ensuremath{\mathbb{Z}}} % Целые числа
\renewcommand\O{\ensuremath{\emptyset}} % Переопределение пустого множества
\newcommand\Q{\ensuremath{\mathbb{Q}}} % Рациональные числа
\newcommand\E{\ensuremath{\mathbb{E}}} % 
\newcommand\V{\ensuremath{\mathbb{V}}} % 
\renewcommand\C{\ensuremath{\mathbb{C}}} % Комплексные числа
\newcommand*\tcircle[1]{%        % Кружок вокруг символа (с помощью TikZ)
  \tikz[baseline=(C.base)]\node[draw,circle,inner sep=0.5pt](C) {#1};\!
}

% --- Для форматирования матриц (черта между столбцами) ---
\makeatletter
\renewcommand*\env@matrix[1][*\c@MaxMatrixCols c]{%
  \hskip -\arraycolsep
  \let\@ifnextchar\new@ifnextchar
  \array{#1}}
\makeatother 

% ----------------------------------------------------
%    Математические операторы и команды
% ----------------------------------------------------
% --- Оператор градиента ---
\newcommand{\grad}{\operatorname{grad}}

% --- Дроби и лимиты в стиле отображения ---
\everymath{\displaystyle}

% --- Системы уравнений ---
\usepackage{systeme}           % Удобный ввод систем уравнений

% --- Лимит с подстрочной стрелкой ---
\let\svlim\lim\def\lim{\svlim\limits}

% --- Короткие обозначения логических операторов ---
\let\implies\Rightarrow
\let\impliedby\Leftarrow
\let\iff\Leftrightarrow
\let\epsilon\varepsilon         % Более традиционное начертание ε

% --- Символ противоречия ---
\usepackage{stmaryrd}
\newcommand\contra{\scalebox{1.5}{$\lightning$}}

% --- Команда для исправлений ---
\definecolor{correct}{HTML}{009900}
\newcommand\correct[2]{\ensuremath{\:}{\color{red}{#1}}\ensuremath{\to }{\color{correct}{#2}}\ensuremath{\:}}
\newcommand\green[1]{{\color{correct}{#1}}}

% --- Горизонтальная линия ---
\newcommand\hr{\noindent\rule[0.5ex]{\linewidth}{0.5pt}}

% --- Скрытие части текста ---
\newcommand\hide[1]{}

% --- Поддержка единиц СИ ---
\usepackage{siunitx}
\sisetup{locale = FR}

% ----------------------------------------------------
%    Окружения для теорем, определений и т.д.
% ----------------------------------------------------
\usepackage{mdframed}
\mdfsetup{skipabove=1em,skipbelow=0em}
\theoremstyle{plain}
\newtheorem{theorem}{Теорема}[chapter]
\newtheorem{lemma}[theorem]{Лемма}
\newcounter{corollaryCounter}[theorem]
\newtheorem{corollary}[corollaryCounter]{Следствие}
\newtheorem*{corollary*}{Следствие}

\theoremstyle{definition}
\newtheorem*{example}{Пример}
\newtheorem*{examples}{Примеры}
\newtheorem*{iexample}{Важный пример}
\newtheorem*{exercise}{Упражнение}
\newtheorem*{definition}{Определение}

\theoremstyle{remark}
\newtheorem*{remark}{Замечание}
\newtheorem*{editremark}{Замечание (от редактора конспекта)}
\newtheorem*{remarks}{Замечания}
\newtheorem*{exerciseAnswer}{Ответ на упражнение}

% --- Упрощённые обозначения ---
\let \thm \theorem
\let \lem \lemma
\let \defn \definition
\let \exmp \example
\let \iex \iexample
\let \exmps \examples
\let \exc \exercise
\let \rem \remark
\let \erem \editremark
\let \rems \remarks
\let \crl \corollary
\let \eans \exerciseAnswer

% --- Математические операторы ---
\DeclareMathOperator{\rank}{rank}
\DeclareMathOperator{\mes}{mes}
\DeclareMathOperator{\diam}{diam}
\DeclareMathOperator{\fix}{fix}
\DeclareMathOperator{\sgn}{sgn}
\DeclareMathOperator{\sign}{sgn}
\DeclareMathOperator{\vp}{v.p.}
\DeclareMathOperator{\Arg}{Arg}
\DeclareMathOperator{\Ln}{Ln}
\DeclareMathOperator{\Arcsin}{Arcsin}
\DeclareMathOperator{\Arccos}{Arccos}
\DeclareMathOperator{\Arctg}{Arctg}
\DeclareMathOperator{\Arcctg}{Arcctg}
\DeclareMathOperator{\Arsh}{Arsh}
\DeclareMathOperator{\Arch}{Arch}
\DeclareMathOperator{\Arth}{Arth}
\DeclareMathOperator{\Arcth}{Arcth}
\DeclareMathOperator{\res}{res}

% --- arctan -> arctg (русский стиль) ---
\renewcommand{\arctan}{\arctg}

% --- Алмазы в конце примеров ---
\usepackage{etoolbox}
\AtEndEnvironment{vb}{\null\hfill$\diamond$}%
\AtEndEnvironment{intermezzo}{\null\hfill$\diamond$}%

% --- Отступы перед теоремами ---
\makeatletter
\def\thm@space@setup{%
  \thm@preskip=\parskip \thm@postskip=0pt
}

% --- Упражнения ---
\newcommand{\oefening}[1]{%
    \def\@oefening{#1}%
    \subsection*{Oefening #1}
}
\newcommand{\suboefening}[1]{%
    \subsubsection*{Oefening \@oefening.#1}
}

% --- Лекции с датами ---
\usepackage{xifthen}
\def\testdateparts#1{\dateparts#1\relax}
\def\dateparts#1 #2 #3 #4 #5\relax{
    \marginpar{\small\textsf{\mbox{#1 #2 #3 #5}}}
}
\def\@lecture{}%
\newcommand{\lecture}[3]{
    \ifthenelse{\isempty{#3}}{%
        \def\@lecture{Лекция #1}%
    }{%
        \def\@lecture{Лекция #1: #3}%
    }%
    \subsection*{\@lecture}
    \marginpar{\small\textsf{\mbox{#2}}}
}

% --- Красивые колонтитулы ---
\usepackage{fancyhdr}
\pagestyle{fancy}
\fancyhead[RO,LE]{\@lecture}
\fancyhead[RE,LO]{}
\fancyfoot[RO,LE]{\thepage}
\fancyfoot[RE,LO]{}
\fancyfoot[C]{\leftmark}

% --- Заметки и исправления в рамке ---
\usepackage{todonotes}
\usepackage{tcolorbox}
\tcbuselibrary{breakable}
\newenvironment{verbetering}{\begin{tcolorbox}[
    arc=0mm,
    colback=white,
    colframe=green!60!black,
    title=Opmerking,
    fonttitle=\sffamily,
    breakable
]}{\end{tcolorbox}}

\newenvironment{noot}[1]{\begin{tcolorbox}[
    arc=0mm,
    colback=white,
    colframe=white!60!black,
    title=#1,
    fonttitle=\sffamily,
    breakable
]}{\end{tcolorbox}}

% --- Поддержка SVG и PDF-графики ---
\usepackage{import}
\usepackage{xifthen}
\usepackage{pdfpages}
\usepackage{transparent}
\newcommand{\incfig}[1]{%
    \def\svgwidth{\columnwidth}
    \import{./figures/}{#1.pdf_tex}
}

% --- Подавление предупреждений при вставке PDF ---
\pdfsuppresswarningpagegroup=1

% --- Автор ---
\author{Belousov M.}
