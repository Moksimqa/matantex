% Some basic packages


\usepackage[utf8]{inputenc}

\usepackage[T2A]{fontenc}
\usepackage{textcomp}

\usepackage[russian]{babel}
\usepackage{url}
\usepackage{graphicx}
\usepackage{float}
\usepackage{tikz}
%\usepackage{pgfplots}
\usepackage{booktabs}
\usepackage{enumitem}
\usepackage{epigraph}
\usepackage{subfiles}
\usepackage{titling}
\usepackage{fancybox}
\usepackage{indentfirst}
\usepackage[left=2cm,right=2cm,top=2cm,bottom=3cm,bindingoffset=0cm]{geometry}
\usepackage{multirow}
\pdfminorversion=7
\usetikzlibrary{decorations.pathreplacing}
% Don't indent paragraphs, leave some space between them
\usepackage{parskip}

% Hide page number when page is empty
\usepackage{emptypage}
\usepackage{subcaption}
\usepackage{multicol}
\usepackage{xcolor}

% Other font I sometimes use.
% \usepackage{cmbright}

% Math stuff
\usepackage{amsmath, amsfonts, mathtools, amsthm, amssymb}
% Fancy script capitals
\usepackage{mathrsfs}
\usepackage{cancel}
% Bold math
\usepackage{bm}
% Some shortcuts
\newcommand\N{\ensuremath{\mathbb{N}}}
\newcommand\R{\ensuremath{\mathbb{R}}}
\newcommand\Z{\ensuremath{\mathbb{Z}}}
\renewcommand\O{\ensuremath{\emptyset}}

\newcommand\Q{\ensuremath{\mathbb{Q}}}
\newcommand\E{\ensuremath{\mathbb{E}}}
\newcommand\V{\ensuremath{\mathbb{V}}}
\renewcommand\C{\ensuremath{\mathbb{C}}}
\newcommand*\tcircle[1]{%
  \tikz[baseline=(C.base)]\node[draw,circle,inner sep=0.5pt](C) {#1};\!
}

%for matrix
\makeatletter
\renewcommand*\env@matrix[1][*\c@MaxMatrixCols c]{%
  \hskip -\arraycolsep
  \let\@ifnextchar\new@ifnextchar
  \array{#1}}
\makeatother

\everymath{\displaystyle}


% Easily typeset systems of equations (French package)
\usepackage{systeme}

% Put x \to \infty below \lim
\let\svlim\lim\def\lim{\svlim\limits}

%Make implies and impliedby shorter
\let\implies\Rightarrow
\let\impliedby\Leftarrow
\let\iff\Leftrightarrow
\let\epsilon\varepsilon

% Add \contra symbol to denote contradiction
\usepackage{stmaryrd} % for \lightning
\newcommand\contra{\scalebox{1.5}{$\lightning$}}

% \let\phi\varphi

% Command for short corrections
% Usage: 1+1=\correct{3}{2}

\definecolor{correct}{HTML}{009900}
\newcommand\correct[2]{\ensuremath{\:}{\color{red}{#1}}\ensuremath{\to }{\color{correct}{#2}}\ensuremath{\:}}
\newcommand\green[1]{{\color{correct}{#1}}}

% horizontal rule
\newcommand\hr{
    \noindent\rule[0.5ex]{\linewidth}{0.5pt}
}

% hide parts
\newcommand\hide[1]{}

% si unitx
\usepackage{siunitx}
\sisetup{locale = FR}

% Environments
\makeatother
% For box around Definition, Theorem, \ldots
\usepackage{mdframed}
\mdfsetup{skipabove=1em,skipbelow=0em}
\theoremstyle{plain}
\newtheorem{theorem}{Теорема}[chapter]
\newtheorem{lemma}[theorem]{Лемма}
\newcounter{corollaryCounter}[theorem]
\newtheorem{corollary}[corollaryCounter]{Следствие}
\newtheorem*{corollary*}{Следствие}

\theoremstyle{definition}
\newtheorem*{example}{Пример}
\newtheorem*{examples}{Примеры}
\newtheorem*{iexample}{Важный пример}
\newtheorem*{exercise}{Упражнение}
\newtheorem*{definition}{Определение}

\theoremstyle{remark}
\newtheorem*{remark}{Замечание}
\newtheorem*{editremark}{Замечание (от редакторов конспекта)}
\newtheorem*{remarks}{Замечания}
\newtheorem*{exerciseAnswer}{Ответ на упражнение}



\let \thm \theorem
\let \lem \lemma
\let \defn \definition
\let \exmp \example
\let \iex \iexample
\let \exmps \examples
\let \exc \exercise
\let \rem \remark
\let \erem \editremark
\let \rems \remarks
\let \crl \corollary
\let \eans \exerciseAnswer



% Математические операторы
\DeclareMathOperator{\rank}{rank}
\DeclareMathOperator{\mes}{mes}
\DeclareMathOperator{\diam}{diam}
\DeclareMathOperator{\fix}{fix}
\DeclareMathOperator{\sgn}{sgn}
\DeclareMathOperator{\sign}{sgn}
\DeclareMathOperator{\vp}{v.p.}
\DeclareMathOperator{\Arg}{Arg}
\DeclareMathOperator{\Ln}{Ln}
\DeclareMathOperator{\Arcsin}{Arcsin}
\DeclareMathOperator{\Arccos}{Arccos}
\DeclareMathOperator{\Arctg}{Arctg}
\DeclareMathOperator{\Arcctg}{Arcctg}
\DeclareMathOperator{\Arsh}{Arsh}
\DeclareMathOperator{\Arch}{Arch}
\DeclareMathOperator{\Arth}{Arth}
\DeclareMathOperator{\Arcth}{Arcth}
\DeclareMathOperator{\res}{res}

%dublicate = 



% arctan -> arctg
\renewcommand{\arctan}{\arctg}

% End example and intermezzo environments with a small diamond (just like proof
% environments end with a small square)
\usepackage{etoolbox}
\AtEndEnvironment{vb}{\null\hfill$\diamond$}%
\AtEndEnvironment{intermezzo}{\null\hfill$\diamond$}%
% \AtEndEnvironment{opmerking}{\null\hfill$\diamond$}%

% Fix some spacing
% http://tex.stackexchange.com/questions/22119/how-can-i-change-the-spacing-before-theorems-with-amsthm
\makeatletter
\def\thm@space@setup{%
  \thm@preskip=\parskip \thm@postskip=0pt
}


% Exercise 
% Usage:
% \oefening{5}
% \suboefening{1}
% \suboefening{2}
% \suboefening{3}
% gives
% Oefening 5
%   Oefening 5.1
%   Oefening 5.2
%   Oefening 5.3
\newcommand{\oefening}[1]{%
    \def\@oefening{#1}%
    \subsection*{Oefening #1}
}

\newcommand{\suboefening}[1]{%
    \subsubsection*{Oefening \@oefening.#1}
}


% \lecture starts a new lecture (les in dutch)
%
% Usage:
% \lecture{1}{di 12 feb 2019 16:00}{Inleiding}
%
% This adds a section heading with the number / title of the lecture and a
% margin paragraph with the date.

% I use \dateparts here to hide the year (2019). This way, I can easily parse
% the date of each lecture unambiguously while still having a human-friendly
% short format printed to the pdf.

\usepackage{xifthen}
\def\testdateparts#1{\dateparts#1\relax}
\def\dateparts#1 #2 #3 #4 #5\relax{
    \marginpar{\small\textsf{\mbox{#1 #2 #3 #5}}}
}

\def\@lecture{}%
\newcommand{\lecture}[3]{
    \ifthenelse{\isempty{#3}}{%
        \def\@lecture{Лекция #1}%
    }{%
        \def\@lecture{Лекция #1: #3}%
    }%
    \subsection*{\@lecture}
    \marginpar{\small\textsf{\mbox{#2}}}
}



% These are the fancy headers
\usepackage{fancyhdr}
\pagestyle{fancy}

% LE: left even
% RO: right odd
% CE, CO: center even, center odd
% My name for when I print my lecture notes to use for an open book exam.
% \fancyhead[LE,RO]{Gilles Castel}

\fancyhead[RO,LE]{\@lecture} % Right odd,  Left even
\fancyhead[RE,LO]{}          % Right even, Left odd

\fancyfoot[RO,LE]{\thepage}  % Right odd,  Left even
\fancyfoot[RE,LO]{}          % Right even, Left odd
\fancyfoot[C]{\leftmark}     % Center

\makeatother




% Todonotes and inline notes in fancy boxes
\usepackage{todonotes}
\usepackage{tcolorbox}

% Make boxes breakable
\tcbuselibrary{breakable}

% Verbetering is correction in Dutch
% Usage: 
% \begin{verbetering}
%     Lorem ipsum dolor sit amet, consetetur sadipscing elitr, sed diam nonumy eirmod
%     tempor invidunt ut labore et dolore magna aliquyam erat, sed diam voluptua. At
%     vero eos et accusam et justo duo dolores et ea rebum. Stet clita kasd gubergren,
%     no sea takimata sanctus est Lorem ipsum dolor sit amet.
% \end{verbetering}
\newenvironment{verbetering}{\begin{tcolorbox}[
    arc=0mm,
    colback=white,
    colframe=green!60!black,
    title=Opmerking,
    fonttitle=\sffamily,
    breakable
]}{\end{tcolorbox}}

% Noot is note in Dutch. Same as 'verbetering' but color of box is different
\newenvironment{noot}[1]{\begin{tcolorbox}[
    arc=0mm,
    colback=white,
    colframe=white!60!black,
    title=#1,
    fonttitle=\sffamily,
    breakable
]}{\end{tcolorbox}}




% Figure support as explained in my blog post.
\usepackage{import}
\usepackage{xifthen}
\usepackage{pdfpages}
\usepackage{transparent}
\newcommand{\incfig}[1]{%
    \def\svgwidth{\columnwidth}
    \import{./figures/}{#1.pdf_tex}
}

% Fix some stuff
% %http://tex.stackexchange.com/questions/76273/multiple-pdfs-with-page-group-included-in-a-single-page-warning
\pdfsuppresswarningpagegroup=1


% My name
\author{Belousov M.}